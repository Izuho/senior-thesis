\begin{eabstract}
\par Quantum computing is currently one of the most active research fields, and a number of companies have developed small to medium-scale (number of qubits to about several hundred) quantum computers called NISQ and made them available in the cloud. Machine learning is also one of the most active research areas today. It is natural that a new research field called quantum machine learning that combines the two emerges. In fact, quantum machine learning was first proposed in the 1980s when research on quantum computing began, and research in this field is expected to become even more active in the future.
\par Two main methods have been proposed in the field of quantum machine learning: the kernel method and the variational method. However, if we try to use the kernel method, it takes $O(N^2)$ computation time where $N$ is the data size. In order to deal with big data such as those used in machine learning, it is considered preferable to adopt the variational method using parameterized gates, which require $O(N)$ computation time for the data size $N$. Although the variational method is faster than the kernel method, it is not enough for large-scale training. Therefore, it is necessary to devise ways to speed up such as parallelization.
\par The variational method is used not only in machine learning but also in optimization and chemistry. A recent study showed how to use multiple quantum processors distributedly in order to perform the variational method for optimization and molecular energy estimation. In this study, the gradient-based algorithm is used, similar to conventional classical machine learning.
\par On the other hand, in quantum machine learning, it is known that the gradient-free algorithm, which analytically optimizes parameters without calculating gradients, tends to perform better than the gradient-based algorithm from the perspective of the convergence rates. A few gradient-free algorithms are proposed now for example, Rotosolve, Rotoselect, Powell, fraxis, etc.
\par We propose a new variational method for quantum machine learning with multiple quantum processors by executing one of the gradient-free algorithms, fraxis. Our work is commendable for devising a new gradient-free parallelizable algorithm to perform the variational method for machine learning, especially for classification. Numerical experiments show that our algorithm can achieve high accuracy in image classification. 
\end{eabstract}