\begin{jabstract}
\par 量子計算は現在最も盛んな研究分野の一つであり、たくさんの会社がNISQと呼ばれる小~中規模(量子ビット数〜数百個程度)の雑音のある量子コンピューターを開発し、クラウド上ではそれらが使用可能になっている。また機械学習もまた、現在最も精力的に研究されている分野の一つである。これら二つのトレンドを組み合わせた量子機械学習とよばれる新しい研究領域は量子計算の研究が始まった1980年台には生まれており、今後ますます盛んになっていくだろう。
\par 量子機械学習の分野では主に二つの方法、カーネル法と変分法が提案されている。しかし、カーネル法を用いようとするとデータサイズに対して$O(N^2$の計算時間がかかってしまう。機械学習に使われるようなビッグデータを扱うためには、データサイズに対して$O(N)$の計算時間で済む、パラメーター化された量子ゲートを用いた変分法を使うのがよいと考えられる。もっとも、変分法はカーネル法よりは早いものの大規模な学習に対してそのまま用いるとやはり時間がかかり過ぎてしまうため、並列化などの高速化の工夫が必要となる。
\par 変分法は機械学習だけではなく最適化や自然科学計算でも用いられる。最新の研究で、複数の量子プロセッサを使って、最適化・分子エネルギー推定を変分法によって分散して行う、新しい高速化の方法が示された。この方法では従来の古典的な機械学習と同じように勾配法を用いて変分法を実行する。
\par 一方で、量子コンピューターを用いた機械学習では勾配法よりも勾配を用いないで解析的にパラメーターを収束させる方が早く収束する傾向があることが知られている。解析的に変分法を実行する方法としてRotosolveやRotoselect, Powell,  fraxisが提案されている。
\par 私たちはfraxisを用いて解析的に変分法を実行することで、複数の量子コンピューターで分散して機械学習を行う方法を提案する。私たちの研究は機械学習のでも特に分類を行うために、解析的に変分法を実行する方法を新しく考案した点で評価される。数値実験を行うことで私たちの方法が画像分類において高い精度を出すことを示す。
\end{jabstract}