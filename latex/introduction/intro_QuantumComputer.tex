\section{Development of Quantum Computer}
\par Classical computers are circuits that process information powered by electricity, and their physical process could be understood by the classical electric laws. On the other hand, when trying to process information using fine particles such as electrons, photons, and atoms, macroscopic classical theory cannot describe the behavior of those fine particles. Here, microscopic quantum theory is needed to describe the behavior of those fine particles. Quantum computers are circuits that perform computations that can only be described by such quantum theory. 

\par Quantum theory developed in the 20-th century, but full-scale research on quantum computers began in the 1980s. Paul Benioff describes the first quantum computer model as a Turing machine in 1980 \cite{Benioff}. In 1988, Y. Yamamoto and K. Igeta proposed the first way of physical realization of a quantum computer \cite{yamamoto}. In the 1990s, many researchers invented the prominent quantum algorithms which can solve certain problems faster than the best-known classical algorithms, which highly encouraged the realization of quantum computers. Famous algorithms like Deutsch–Jozsa algorithm \cite{deutsch}, Shor's algorithm \cite{shor}, Grover's algorithm \cite{grover} were invented in this period. 

\par In 1998 I. Chuang, N. Gershenfeld and M. Kubinec created the first 2-qubit quantum computer to perform Grover’s algorithm \cite{2quFirst}. After that, over the next 20 years many types of quantum computers have been realized. Now even some quantum computers are available in the cloud. Photonic quantum computers developed by Xanadu, trapped ion quantum computers by IonQ, superconductiong quantum computers by Rigetti, etc. are now available in Amazon Web Services (AWS) \cite{AWS}. IBM Quantum also delivers the service of its quantum computers, simulators and quantum-classical hybrid environments via the cloud \cite{ibmq}.

\par The key concept of quantum computers is a qubit or a quantum bit, the quantum version of the classical binary bit. A qubit simultaneously takes two states, which is peculiar to quantum mechanics. Furthermore, multiple qubits can exhibit quantum entanglement which expresses higher correlation between qubits than is observed between the classical bits. These two properties are fundamental for quantum computing.

\par These quantum computers today are called NISQ for Noisy Intermediate-Scale Quantum devices. These devices have only a limited number (dozens to hundreds) of qubits that necessarily cannot be entangled with each other. These devices also cannot correct the error occurred during the implementation time, which means these devices can produce only the approximate outcomes.
