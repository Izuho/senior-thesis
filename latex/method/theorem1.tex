\subsubsection{Theorem}
When the matrix $G=\sum_{i=1}^Ny_iG_i$ has eigenvalues $\lambda_1 \leq \lambda_2 \leq \lambda_3$ (corresponding
eigenvectors as $n_1, n_2, n_3$, $\|n_1\|= \|n_2\| = \|n_3\| = 1$), then the followings holds.
$$\argmax_{\|n\|=1} n^\top Gn = n_3$$
$$\max_{\|n\|=1} n^\top Gn = \lambda_3$$
\subsubsection{Proof}
We define the function $f$ and $\hat{\bm{n}}$:
$$f(n,\tau)=n^\top Gn-\tau(\|n\|^2-1)$$
$$\hat{n} = \argmax_{\|n\|=1} n^\top Gn$$

By the method of Lagrange multipliers, at the point $\hat{n}$, there exists 
$\hat{\tau}$ that satisfies the following.
$$\frac{\partial f}{\partial n}(\hat{n}, \hat{\tau} ) = 0$$
By solving these equations, we get
$$2G \hat{n} = \hat{\tau} \hat{n}$$
Here we use the formula $\frac{\partial }{\partial \bm{x}}\bm{x}^\top A\bm{x}=(A+A^\top)\bm{x}$ (see \ref{chap:appendix}).

\par Therefore, $\hat{n}$ is one of the eigenvectors of $G$, $n_1, n_2, n_3$.
Since 
$$\begin{aligned}
n_k^\top Gn_k&=\lambda_kn_k^\top n_k\\
&=\lambda_k\|n_k\|^2_2\\
&=\lambda_k
\end{aligned}$$
holds, from the condtion $\lambda_1\leq\lambda_2\leq\lambda_3$  directory follows $\hat{n}=n_3$.